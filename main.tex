% a mashup of hipstercv, friggeri and twenty cv
% https://www.latextemplates.com/template/twenty-seconds-resumecv
% https://www.latextemplates.com/template/friggeri-resume-cv

\documentclass[lighthipster]{simplehipstercv}
% available options are: darkhipster, lighthipster, pastel, allblack, grey, verylight, withoutsidebar
% withoutsidebar
\usepackage[utf8]{inputenc}
\usepackage[default]{raleway}
\usepackage[margin=1cm, a4paper]{geometry}



%------------------------------------------------------------------ Variablen

\newlength{\rightcolwidth}
\newlength{\leftcolwidth}
\setlength{\leftcolwidth}{0.23\textwidth}
\setlength{\rightcolwidth}{0.75\textwidth}

%------------------------------------------------------------------
\title{First CV}
\author{\LaTeX{} Asad}
\date{May 2024}

\pagestyle{empty}
\begin{document}


\thispagestyle{empty}
%-------------------------------------------------------------

\section*{Start}

\simpleheader{headercolour}{}{Asad Arshad}{Astrophysicist \& Programmer}{white}



%------------------------------------------------

% this has to be here so the paracols starts..
\subsection*{}
\vspace{4em}

\setlength{\columnsep}{1.5cm}
\columnratio{0.23}[0.75]
\begin{paracol}{2}
\hbadness5000
% \backgroundcolor{c[1]}[rgb]{1,1,0.8} % cream yellow for column-1 %\backgroundcolor{g}[rgb]{0.8,1,1} % \backgroundcolor{l}[rgb]{0,0,0.7} % dark blue for left margin

\paracolbackgroundoptions

% 0.9,0.9,0.9 -- 0.8,0.8,0.8


\footnotesize
{\setasidefontcolour
\flushright
% \begin{center}
%     \roundpic{jack.jpg}
% \end{center}
% \bigskip

\vspace{4.5em}
\bg{cvgreen}{white}{About me}\\[0.5em]

{\footnotesize
An inquisitive, ambitious and creative student, with a passion of learning, always seeking new ways to 
understand the cosmos, while experiencing and enjoying the world of science, research and technology, as well as have \emph{fun}.
}
\bigskip

% \vspace{1.5cm}

\bg{cvgreen}{white}{Personal} \\[0.5em]
Gender: Male \\
Nationality: Pakistani \\
circa 2002 AD\\

\bigskip

\bg{cvgreen}{white}{Areas of specialization} \\[0.5em]

•~Astrophysics \\~•~Scientific Research\\~•~Exoplanetary Science\\~•~ GIS \& Remote Sensing \\~•~Programming\\~•~Scientific \& Fictional Writing

\bigskip


\bg{cvgreen}{white}{Interests}\\[0.5em]

Reading, Writing, Programming, Cycling \\and Listening to the Music during all these.

\bigskip

\vspace{3em}

\bg{cvgreen}{white}{Application Softwares}\\[0.5em]

•~Jupyter Notebooks \\
•~Visual Studio Code \\
•~FITS Liberator \\
•~Microsoft Office \\
•~ArcGIS Pro \& Desktop \\
•~QGIS \\
•~Google Earth Engine \& Google Earth Pro \\
•~Stellarium \\

\bigskip

\vspace{8em}
\bigskip

\infobubble{\faWhatsapp}{cvgreen}{white}{+92-349-4906282}
\infobubble{\faLinkedin}{cvgreen}{white}{\href{https://www.linkedin.com/in/asad-arshad-b42a0a225}{Asad Arshad}}
\infobubble{\faGithub}{cvgreen}{white}{\href{https://github.com/def-fun7}{def\_fun}}
\infobubble{\faTwitter}{cvgreen}{white}{\href{https://x.com/c299792458_}{@c299792458\_}}
\infobubble{\faEnvelope}{cvgreen}{white}{\href{mailto:asad.mail@tutamail.com}{asad.mail}}
\infobubble{\faUser}{cvgreen}{white}{\href{https://def-fun7.github.io/Portfolio/}{My Portfolio}}

\phantom{turn the page}

\phantom{turn the page}
}
%-----------------------------------------------------------
\switchcolumn

\small
\section*{On Going Projects }

\begin{tabular}{r| p{0.5\textwidth} c}
    
    \cvevent{2025--Pre}{Down the Black Hole: A wobbly web experience}{Lead}{HTML, CSS, JS \color{cvred}}
    {As for my Bachelors final year project, I along with a friend, are currently designing a webpage that will contain information, articles, current research, illustrations from across the web, links as well as basic help to navigate data regrading BHs from sources like NASA and the things we learn as we venture in this rabbit hole. }{bh.jpg}
\end{tabular}
\vspace{2em}

% \begin{minipage}[t]{0.35\textwidth}
% \section*{Degrees}
% \begin{tabular}{r p{0.6\textwidth} c}
%     \cvdegree{2019--21}{Intermediate}{F. Sc (Pre Eng) [A+]}{ \newline Govt. College Township \color{headerblue}}{}{gct.jpeg} \\
%     \cvdegree{2021--25}{Bachelors }{Space Sciences [3.86]}{ \newline Dept. of Space Science \newline University of The Punjab \color{headerblue}}{}{pu.png} \\
% \end{tabular}
% \end{minipage}\hfill
% \begin{minipage}[t]{0.3\textwidth}
% \section*{Programming}
% \begin{tabular}{r @{\hspace{0.5em}}l}
%      \bg{skilllabelcolour}{iconcolour}{python} & \barrule{0.55}{0.5em}{cvpurple} \\ 
%     %  \bg{skilllabelcolour}{iconcolour}{R} & \barrule{0.2}{0.5em}{cvpurple} \\
%      \bg{skilllabelcolour}{iconcolour}{javascript} & \barrule{0.43}{0.5em}{cvpurple} \\
%      \bg{skilllabelcolour}{iconcolour}{html, css} &  \barrule{0.45}{0.5em}{cvpurple}\\
%      \bg{skilllabelcolour}{iconcolour}{\LaTeX} & \barrule{0.5}{0.5em}{cvpurple} \\
% \end{tabular}
% \centering\small{(\LaTeX~compiled on \today)}
% \end{minipage}

\normalsize

\section*{Past Work}
\begin{tabular}{r| p{0.5\textwidth} c}
    \cvevent{2024}{Studying the effects of ENSO on Moonsoon in Pakistan}{Thought Leader}{Google Earth Engine \color{cvred}}
    {Our aim was to use EE datasets to calculate \textbf{Oceanic Nino Index (ONI)} and \textbf{Standarized Precepitation Index (SPI)} to see how the rainfall in moonsoon season in Punjab, Pakistan is effected by El Nino and La Nina.
    }{earthengine.jpeg} \\
    \cvevent{2024}{QSOs Spectra and Virial BH masses on SDSS}{Solo}{SciServer \color{cvred}}
    {In this little side project, I explored the vast dataset on Spectra of QSOs through python as well as performed some stats on the BH mass dataset from \textbf{Vizier}.}{ss.png} \\
    \cvevent{2023}{Exoplanets around ``TRAPPIST-1'' through ``KEPLER''}{Lead}{FITS Liberator \& MAST \color{cvred}}{Analysing the Light curves of ``TRAPPIST-1'' that we made using the ``Transit data'' from \textbf{Kepler Mission} and \textbf{K2} and studied the orientation and general planetary parametes of the planets in TRAPPIST system.}{mast.jpeg} \\
    \cvevent{2023}{Satellite Orbits and Ground Track using ``Keplerian Elements}{Solo}{Python \& VS code \color{cvred}}{We used the keplerian elements information on Satellites like Landsat and wrote a python script that calculated the path of satellite and animated the plot as well as it's ground track on a longitude and latitude axes with labels.}{py.jpg} \\
\end{tabular}

\begin{center}
    \small
    \textbf{and many many more relating to orbital mechanics and mostly in GIS, Remote Sensing and Maps.}
\end{center}
\vspace{2em}

\begin{minipage}[t]{0.3\textwidth}
\section*{Positions Held}
\begin{tabular}{>{\footnotesize\bfseries}r >{\footnotesize}p{0.55\textwidth}}
    2024 & Summer Internship at \emph{PMD, HQ, Islamabad}\\
    2024 & Dr.~Khalid's assistant in \emph{SARNET} course\\
\end{tabular}
\bigskip

\section*{Languages}
\begin{tabular}{l | ll}
\textbf{Urdu} & C2 & {\phantom{x}\footnotesize mother tongue} \\
\textbf{English} & C2 & \pictofraction{\faCircle}{cvgreen}{4}{black!30}{1}{\tiny} \\
% \textbf{Spanish} & C2 & \pictofraction{\faCircle}{cvgreen}{1}{black!30}{3}{\tiny} \\
% \textbf{Italian} & C2 & \pictofraction{\faCircle}{cvgreen}{3}{black!30}{1}{\tiny}
\end{tabular}
\bigskip

\end{minipage}\hfill
\begin{minipage}[t]{0.3\textwidth}
\section*{Programs}
\begin{tabular}{>{\footnotesize\bfseries}r >{\footnotesize}p{0.7\textwidth}}
    2024 & \emph{ONI using OISST v2.1 dataset}, Google Earth Engine. \\
    2024 & \emph{Orbits \& Keplerian Elements in Python}, Google Colab.
\end{tabular}
\bigskip

\section*{Writings}
\begin{tabular}{>{\footnotesize\bfseries}r >{\footnotesize}p{0.6\textwidth}}
    Aug. 2021 & ``Would you like an Omelete?''.
\end{tabular}
\end{minipage}






\vfill{} % Whitespace before final footer

%----------------------------------------------------------------------------------------
%	FINAL FOOTER
%----------------------------------------------------------------------------------------
\setlength{\parindent}{0pt}
\begin{minipage}[t]{\rightcolwidth}
\begin{center}\fontfamily{\sfdefault}\selectfont \color{black!70}
{\small Asad Arshad \icon{\faEnvelopeO}{cvgreen}{} \protect\href{mailto:asad.mail!tutamail.com}{asad.mail@tutamail.com} \icon{\faMapMarker}{cvgreen}{} Lahore \icon{\faPhone}{cvgreen}{} +92/349 4906282 
% \newline\icon{\faAt}{cvgreen}{} \protect\url{jack@sparrow.com}
}
\end{center}
\end{minipage}

\end{paracol}

\end{document}
